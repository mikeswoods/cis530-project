%%%%%%%%%%%%%%%%%%%%%%%%%%%%%%%%%%%%%%%%%
%
% CIS530 final report - Michael Woods and Stuart Wagner
%
% Arsclassica Article
% LaTeX Template
% Version 1.1 (10/6/14)
%%%%%%%%%%%%%%%%%%%%%%%%%%%%%%%%%%%%%%%%%

%-------------------------------------------------------------------------------
%	PACKAGES AND OTHER DOCUMENT CONFIGURATIONS
%-------------------------------------------------------------------------------

\documentclass[
10pt, % Main document font size
a4paper, % Paper type, use 'letterpaper' for US Letter paper
oneside, % One page layout (no page indentation)
%twoside, % Two page layout (page indentation for binding and different headers)
headinclude,footinclude, % Extra spacing for the header and footer
BCOR5mm, % Binding correction
]{scrartcl}
\usepackage{fancyvrb}
%%%%%%%%%%%%%%%%%%%%%%%%%%%%%%%%%%%%%%%%%
% Arsclassica Article
% Structure Specification File
%
% This file has been downloaded from:
% http://www.LaTeXTemplates.com
%
% Original author:
% Lorenzo Pantieri (http://www.lorenzopantieri.net) with extensive modifications by:
% Vel (vel@latextemplates.com)
%
% License:
% CC BY-NC-SA 3.0 (http://creativecommons.org/licenses/by-nc-sa/3.0/)
%
%%%%%%%%%%%%%%%%%%%%%%%%%%%%%%%%%%%%%%%%%

%----------------------------------------------------------------------------------------
%	REQUIRED PACKAGES
%----------------------------------------------------------------------------------------

\usepackage[
nochapters, % Turn off chapters since this is an article        
beramono, % Use the Bera Mono font for monospaced text (\texttt)
eulermath,% Use the Euler font for mathematics
pdfspacing, % Makes use of pdftex’ letter spacing capabilities via the microtype package
dottedtoc % Dotted lines leading to the page numbers in the table of contents
]{classicthesis} % The layout is based on the Classic Thesis style

\usepackage{arsclassica} % Modifies the Classic Thesis package

\usepackage[T1]{fontenc} % Use 8-bit encoding that has 256 glyphs

\usepackage[utf8]{inputenc} % Required for including letters with accents

\usepackage{graphicx} % Required for including images
\graphicspath{{Figures/}} % Set the default folder for images

\usepackage{enumitem} % Required for manipulating the whitespace between and within lists

\usepackage{lipsum} % Used for inserting dummy 'Lorem ipsum' text into the template

\usepackage{subfig} % Required for creating figures with multiple parts (subfigures)

\usepackage{amsmath,amssymb,amsthm} % For including math equations, theorems, symbols, etc

\usepackage{varioref} % More descriptive referencing

%----------------------------------------------------------------------------------------
%	THEOREM STYLES
%---------------------------------------------------------------------------------------

\theoremstyle{definition} % Define theorem styles here based on the definition style (used for definitions and examples)
\newtheorem{definition}{Definition}

\theoremstyle{plain} % Define theorem styles here based on the plain style (used for theorems, lemmas, propositions)
\newtheorem{theorem}{Theorem}

\theoremstyle{remark} % Define theorem styles here based on the remark style (used for remarks and notes)

%----------------------------------------------------------------------------------------
%	HYPERLINKS
%---------------------------------------------------------------------------------------

\hypersetup{
%draft, % Uncomment to remove all links (useful for printing in black and white)
colorlinks=true, breaklinks=true, bookmarks=true,bookmarksnumbered,
urlcolor=webbrown, linkcolor=RoyalBlue, citecolor=webgreen, % Link colors
pdftitle={}, % PDF title
pdfauthor={\textcopyright}, % PDF Author
pdfsubject={}, % PDF Subject
pdfkeywords={}, % PDF Keywords
pdfcreator={pdfLaTeX}, % PDF Creator
pdfproducer={LaTeX with hyperref and ClassicThesis} % PDF producer
} 

\hyphenation{Fortran hy-phen-ation}

%-------------------------------------------------------------------------------
% TITLE AND AUTHOR(S)
%-------------------------------------------------------------------------------

\title{\normalfont\spacedallcaps{CIS530 Final Report}}

\author{\spacedlowsmallcaps{Stuart Wagner \& Michael Woods}}

\date{}

\begin{document}

%-------------------------------------------------------------------------------
% HEADERS
%-------------------------------------------------------------------------------

% The header for all pages (oneside) or for even pages (twoside)
\renewcommand{\sectionmark}[1]{\markright{\spacedlowsmallcaps{#1}}}

% Uncomment when using the twoside option - this modifies the header on
% odd pages
%\renewcommand{\subsectionmark}[1]{\markright{\thesubsection~#1}} 
\lehead{\mbox{\llap{\small\thepage\kern1em\color{halfgray} \vline}\color{halfgray}\hspace{0.5em}\rightmark\hfil}}

 % Enable the headers specified in this block
\pagestyle{scrheadings}

%-------------------------------------------------------------------------------

\maketitle

\setcounter{tocdepth}{2}

\tableofcontents % Print the table of contents

%-------------------------------------------------------------------------------
% Introduction
%-------------------------------------------------------------------------------

\section{Introduction}

Text classification on the basis of high or low information density has until
recently
\footnote{Nenkova \& Yang ``Detecting Information-Dense Texts in Multiple News Domains'' 2014 \newline 
\url{http://www.aaai.org/ocs/index.php/AAAI/AAAI14/paper/view/8430/8622}} 
been a relatively overlooked task in the field of Computational Linguistics. 
Building off the contributions of recent research, we seek to improve accuracy on
correctly classifying documents with high information density using the 
tools and techniques of machine learning.

Information density classification has many practical applications, most of
which relate to the task of automatic summarization. Text summarization has
relied partially on the use of KL divergence---the measurement of a difference
between one probability distribution and another---in order to determine the
quality of a generated summary. However, this measure fails to account for
information density, a rather abstract metric of the ``informativeness'' of a
given piece of writing.

Nenkova and Yang used a number of features in developing their model. We used
many similar features, however as students new to NLP, we instead thought that
using a variety of learners could potentially provide better results. We
therefore approached the problem as a machine learning problem that leveraged
our limited knowledge of NLP for the purposes of feature engineering. Could we
learn which features were more correlated with information density? Could we
leverage a variety of models, using ensemble learning, to improve prediction
accuracy? The answer to those questions, generally speaking, is yes.

%-------------------------------------------------------------------------------
% Methods and Resources
%-------------------------------------------------------------------------------

\section{Methods and Resources}

We decided to frame the problem of classifying a given article as either 
information dense or sparse as primarily a machine learning problem. As with
most problems that can be addressed with the methodologies of machine learning,
having some knowledge of the domain--in this case text classification--can 
provide useful insights when selecting and extract features from raw data. With 
the approaches we learned in CIS530 regarding the analysis of the structural
components of text, along with our own intuitions and experimental 
insights, we outline the various features, tooling, and resources we employed 
for the project below.

\subsection{Data}

Apart from the $2,282$ training and $250$ test leads we were provided for the 
project, we made use of the following third party data resources:

% === MRC Psycholinguistic Database ===
\paragraph{\textbf{MRC Psycholinguistic Database}}
\hfill \newline \noindent \url{http://websites.psychology.uwa.edu.au/school/MRCDatabase/uwa_mrc.htm}
\hfill \newline \noindent The MRC Psycholinguistic Database is a machine readable
dictionary containing a number of linguistic and psycholinguistic
attributes for each word \footnote{\url{http://www.psych.rl.ac.uk/}}. 
Specifically, we primarily made use of the 4,923 word base lexicon included in 
the MRC distribution.

% === Linguistic Inquiry and Word Count (LIWC) ===
\paragraph{\textbf{Linguistic Inquiry and Word Count (LIWC)}}
\hfill \newline \noindent \url{http://www.liwc.net}
\hfill \newline \noindent The Linguistic Inquiry and Word Count (LIWC) is a text
analysis software program designed to calculate the degree to ``which people use 
different categories of words across a wide array of texts, including emails, 
speeches, poems, or transcribed daily speech.'' \footnote{\url{http://www.liwc.net/}}

\subsection{Tools and Software Packages}

% === Stanford CoreNLP ===
\paragraph{\textbf{Stanford CoreNLP}}
\hfill \newline \noindent \url{http://nlp.stanford.edu/software/corenlp.shtml}
\hfill \newline \noindent A suite of text processing tools capable of part-of-
speech (POS) tagging, named entity recognition (NER), tokenization,
lemmatization, sentiment analysis, as well as dependency relation and 
syntactic parse extraction.

% === scikit-learn ===
\paragraph{\textbf{scikit-learn}}
\hfill \newline \noindent \url{scikit-learn http://scikit-learn.org/}
\hfill \newline \noindent A popular Python machine learning framework with
implementations for a large selection of popular classifiers.

% === NumPy ===
\paragraph{\textbf{NumPy}}
\hfill \newline \noindent \url{http://www.numpy.org}
\hfill \newline \noindent A scientific computing package for Python providing a
native n-dimensional array datatype, as well as various linear algebra,
statistical, and general mathematical functions suitable for ``number
crunching.''

% === NLTK ===
\paragraph{\textbf{NLTK}}
\hfill \newline \noindent \url{http://www.nltk.org/} 
\hfill \newline \noindent A Python library for working with natural 
language that includes a number of convenient text processing tools and built 
in corpora for multiple languages.

%------------------------------------------------

\subsection{Features}

Building off the work of Nenkova and Yang, we leveraged a number of similar 
features in our approach to the problem of textual information density
classification.

% === CoreNLP word count--binary bag-of-words ===
\paragraph{\textbf{(1) CoreNLP word count--binary bag-of-words}}
\hfill \newline \noindent \textit{9799 dimensions}. All unique tokens occurring
between \texttt{<word></word>} tags in the generated XML obtained from running
all training files through the CoreNLP tool were collected into a list
and converted to lowercase. Any tokens consisting of stop words, as defined by
NLTK's English stop word list or punctuation characters were removed. The list
was then sorted alphabetically in ascending order and each word was assigned an
index. To construct the feature vector for a given lead, the text of the lead
was processed in a manner similar to that of the XML output and a vector the
size of the bag-of-words model was initialized with zeros. If the lead contained
the $i$-th word in the bag-of-words model, the $i$-th position in the feature
vector was set to $1$.

% === CoreNLP sentence information ===
\paragraph{\textbf{(2) CoreNLP sentence information}} 
\hfill \newline \noindent \textit{7 dimensions}. To construct the feature
vector, each processed lead had a number of pieces of information extracted from
it directly or from the XML file generated from it using CoreNLP. These
quantities define the features indicated below along with our prediction on it’s
impact on information density ($+$/$-$):
\begin{itemize}
	\item The number of sentences per lead. ($-$)
	\item The number of non-unique tokens per lead. ($-$)
	\item The number of named entities recognized in the lead. ($+$)
	\item The number of nouns in the lead (tokens tagged as \texttt{NN}, \texttt{NNS}, \texttt{NNP}, or \texttt{NNPS}) ($+$)
	\item The number of words with over six letters. ($+$)
	\item The number of quotation marks appearing in the lead. ($+$)
	\item The aggregate absolute sentiment score of the lead. Each sentence was assigned a numeric score according to the following rubric: ``very negative'': $2$, ``negative'': $1$, ``neutral'': $0$, ``positive'': $1$, and ``very positive'': $2$. The scores were then summed and the absolute value was taken. ($-$)
\end{itemize}

% === LIWC ===
\paragraph{\textbf{(3) LIWC}}
\hfill \newline \noindent \textit{78 dimensions}. Our hypothesis was that words
indicating sentiment and emotions might indicate sentences that were less
information dense. We thought these might be positively correlated with
information density: present tense, numbers, impersonal pronouns, and causation.
We also believed multiple attributes may be negatively correlated, including:
1st person usage, auxillary verbs and past tense, quantifiers, and positive and
negative emotion. For a listing of the data points in the LIWC, refer to the
table at \url{http://www.liwc.net/comparedicts.php}.

% === CoreNLP production rule count--binary bag-of-words ===
\paragraph{\textbf{(4) CoreNLP production rule count--binary bag-of-words}}
\hfill \newline \noindent \textit{5654 dimensions}. The syntactic parse
generated by CoreNLP for each sentence from a given lead was extracted from the
corresponding \texttt{<parse></parse>} tags in the corresponding XML file. A
parse tree was constructed in the same manner as described in Homework 3, where
only non-terminal nodes were considered. To construct the master list of
production rules, the outlined process was repeated for every lead file in the
training set and every unique production rule was collected into a list. Each
rule was then assigned an index, $0$ to $N-1$, where N is the number of unique
production rules. To construct a feature vector for a given lead, a $1 x N$
vector of zeros was created and every production rule occurring in the sentences
of the lead was extracted. For every rule in the master list, if the $i$-th rule
occurred in the rules extracted from the lead, the $i$-th index in the feature
vector was set to $1$. The belief was that perhaps certain production rules may
be related with information density.

% === LIWC ===
\paragraph{\textbf{(5) MRC dictionary--binary bag-of-words}}
\hfill \newline \noindent \textit{4923 dimensions}. 
The MRC Psycholinguistic database word dictionary consisting of $4,923$ words
was used to construct a binary bag-of-words model in a similar manner to the
CoreNLP binary bag-of-words model. We also attempted counts normalized by length
of lead, however there did not appear to be a significant difference in
accuracy. The text of each lead was tokenized, converted to lowercase and
stripped of stop words. The benefit of using this approach is that the
dictionary is independent of the task and training data (Nenkova and Yang).

% === Dependency relations--binary bag-of-words ===
\paragraph{\textbf{(6) Dependency relations--binary bag-of-words}}
\hfill \newline \noindent \textit{5000 dimensions}. 
For each lead file in the training set, the <dep></dep> nodes were extracted 
from the corresponding CoreNLP XML files and converted to a 3-tuple consisting 
of the dependency relation type, the governor word, and the dependent word. 
Each tuple instance was counted, and the top 5000 dependency tuples were used 
to build a bag-of-words vector in much the same manner as features (1), (4)
and (5). Constructing the feature vector for a given lead proceeded in much the
same manner as (5) as well. Like (4), we believed that certain dependency
relations are significantly more prevalent in information dense articles than
sparse ones.

%------------------------------------------------

\subsection{Models}
Regarding the actual task of classification, we constructed several 
classification models using the features described above. The model 
configurations are as follows:

\begin{itemize}
	\item Logistic regression with a L2 penalty and class weights $1$: $0.58$ and
	$-1$: $0.42$ set to reflect the true distribution of labels in the test set.

	\item Linear SVM with L2 loss and penalty functions and class 
	weights $1$: $0.58$ and $-1$: $0.42$ set to reflect the true distribution of
	labels in the test set.

	\item Bernoulli naive bayes with default parameters.

	\item Ensemble method with the three classifiers described above. Each 
	classifier makes a prediction for a given observation and a majority vote is
	taken to determine the final predicted label.
\end{itemize}

%-------------------------------------------------------------------------------
% Final system
%-------------------------------------------------------------------------------

\section{Final system}

\subsection{Team}

We are team ``$W^2$,'' or as it appears on the leaderboard in ASCII,
\texttt{W\string^2}.

\subsection{System Design}

From the very onset of the project, we exercised great discipline in producing
what we believe to a clean, extensible system for making predictions about text.
The software could easily have been treated as disposable---and designed as such
---since the predictions produced by the system are ultimately the real value
and not the software itself. Our rationale for putting a large, up-front effort
into properly designing a data pipeline was to introduce a lower mental overhead
cost when attempting to quickly evaluate the effectiveness of various
featuresets and models as the project progressed. The effort paid off many times
over, as creating new models and features was the same as defining a new Python
module. The model or feature could then later be referred to by name and
manipulated through set of generic functions.

\subsection{Models}

All models are implemented as regular Python modules in the
\texttt{project.models} package in the project \texttt{code} directory.
Additionally, all model modules are expected to define the functions described
below, as this provides a common interface to interact with the model
implementation: 
\begin{Verbatim}[frame=single]
def preprocess(model_name, train_files, test_files)
def train(model_name, train_files,  labels, *args, **kwargs)
def predict(model_name, model_obj, test_files, *args, **kwargs)
\end{Verbatim}

\subsubsection{Lookup}
\noindent A module can then easily be looked up by name like so:
\begin{Verbatim}[frame=single]
linear_model = get_model('linear_model')
\end{Verbatim}
\noindent which is defined in the following simple function which makes use
of the \texttt{importlib} from the Python standard libary:
\begin{Verbatim}[frame=single]
def get_model(name):
    return importlib.import_module('project.models.'+name)
\end{Verbatim}

\subsection{Features}

\noindent Much like models, features are also implemented as Python modules, as
this allows for quick prototyping and iterative refinement of new feature types.
\\

\noindent In a similar manner to models described above, features (or more
accurately feature-izers) can easily be resolved with the following function:

\begin{Verbatim}[frame=single]
def get_feature(name):
    return importlib.import_module('project.features.'+name)
\end{Verbatim}

\noindent Features can also expose the following functions:

\begin{Verbatim}[frame=single]
def build(feature_name, *args, **kwargs)
def featurize(feature_name, feature_obj=None, *args, **kwargs)
\end{Verbatim}

\noindent where \texttt{build} performs the initial setup work needed to
featurize an input, and \texttt{featurize} performs the actual conversion of an
input into a feature vector.

\break
\subsection{Pipeline}

The conceptual classification pipeline is outlined below in pseudocode, which
describes the general flow of information in the system, initially from raw
input text to final predicted labels:

\begin{Verbatim}[frame=single]
TRAIN_FILES = framework.get_train_files()

if in-submission-mode:
    TEST_FILES = framework.get_test_files()
else:
    TEST_FILES = TRAIN_FILES

# if the model defines an optional preprocess() function, 
# call it:
if is_preprocess_defined('foo')
    optional = preprocess('foo', TRAIN_FILES, TEST_FILES)
else:
    optional = []

model_obj = train('foo', TRAIN_FILES, TRAIN_LABELS, *optional)

return predict('foo', model_obj, TEST_FILES, *optional)
\end{Verbatim}

\subsection{Testing}

Finally, the above pipeline is invoked from within the context of a testing
framework we designed. The framework itself provides the input files to the 
prediction pipeline, which may vary depending on whether an actual prediction
submission is being generated. If a leaderboard submission is to be generated, 
then actual test data is supplied to the pipeline by the framework, otherwise 
some holdout from the training data is used as test data. After the predictions 
have been made, the framework provides basic reporting information regarding 
the accuracy, precision, and F-score of the results.

%-------------------------------------------------------------------------------
% Outcomes
%-------------------------------------------------------------------------------

\section{Outcomes}

All experiments were conducted using the \texttt{scikit-learn} 
\texttt{cross\_validation.train\_test\_split}, function with $10\%$ of
the training data held out as the test set. \\

\noindent All submission configurations and related results are 
summarized in the table below. All feature sets are referenced by number as 
given in section 2.3.

\subsection{Submission results}
\begin{center}
    \begin{tabular}{| c | c | l | l|}
    \hline
    Attempt & Accuracy & Model               & Feature sets    \\ \hline
    1       & $75.6\%$ & Logistic regression  & $1$, $2$, $3$  \\ \hline
    2       & $76.0\%$ & Logistic regression /w PCA(n=500)+gridsearch &  $2$, $3$, $4$, $5$ \\ \hline
    3       & $78.4\%$ & Ensemble classifier (LR+SVM+NB) &$2$, $3$, $4$, $5$       \\ \hline
    4       & $78.4\%$ & Ensemble classifier (LR+SVM+NB) & $1$, $2$, $3$, $4$, $5$ \\ \hline
    5       & $79.6\%$ & Ensemble classifier (LR+SVM+NB) & $2$, $3$, $4$, $5$, $6$ \\ \hline
    \end{tabular}
\end{center}

%-------------------------------------------------------------------------------
% Discussion of Results and Conclusion
%-------------------------------------------------------------------------------
\section{Discussion of Results and Conclusion}

Overall, we are quite pleased with our results. Each test submission resulted in
either the same or a higher level of predictive accuracy. We believe this is due
to the design of our system which allowed for quick a test--prototype--report
cycle of various model and feature configurations, which in turn encouraged
experimentation.

\subsection{Analysis of feature performance}
All told, we developed seven different feature sets: 
\begin{enumerate}
	\item \texttt{[CoreNLP word count;binary bag-of-words]} 
	\item \texttt{[CoreNLP word count;TDIDF normalized bag-of-words]} 
	\item \texttt{[CoreNLP sentence information]} 
	\item \texttt{[LIWC]}
	\item \texttt{[CoreNLP production rule count;binary bag-of-words]}
	\item \texttt{[MRC dictionary;binary bag-of-words]}
	\item \texttt{[Dependency relations;binary bag-of-words]}
\end{enumerate}

\subsubsection{The good}
Of the above features, \texttt{[CoreNLP sentence information]},
\texttt{[LIWC]}, \texttt{[CoreNLP production rule count;binary bag-of-words]}, 
\texttt{[MRC dictionary;binary bag-of-words]}, 
and \texttt{[Dependency relations:binary bag-of-words]} performed very well,
usually achieving a prediction accuracy in the $80-85\%$ range when each
featureset was used to make predictions in isolation with a logistic regression
classifier. When used together in the ensemble model, the predictive accuracy
spanned $80-89\%$ when running against a cross-validated test set.

\subsubsection{The not-so-good}
While the performance of the featureset \texttt{[CoreNLP word count;binary bag-
of-words]} was not terrible, it was not great either. When used as a feature set
in isolation with a logistic regression classifier, prediction accuracies
generally failed to break $75\%$. This could be due to the fact that a binary
bag-of-words model regarding the distribution of information dense/sparse
articles is simply just not that informative, meaning the words that appeared in
information dense and information sparse articles were not different enough to
act as a good discriminative metric.

\subsubsection{The bad}
The \texttt{[CoreNLP word count;TDIDF normalized bag-of-words]} featureset
consistently performed the worst, routinely failing to break $70\%$ during
testing and achieving an average perdiction accuracy in the $60-65\%$ range. The
poor performance of this featureset was probably due to similar reasons as that
of the \texttt{[CoreNLP word count;binary bag-of-words]} featureset, combined
with a potential error in the implementation.

\subsection{Keys to success}

The ensemble model, we believe, allowed our system to outperform most other 
teams, even when using similar feature sets. The intuition behind the success 
of such an approach is straightforward: when in doubt, seek multiple opinions in
order to come to an answer. The trade-offs when using an ensemble approach to
prediction are mostly related to time--instead of training a single classifier,
three, five, etc. classifiers need to be trained instead. Also, it is necessary
to use an odd number of classifiers that achieve roughly the same level of
predictive accuracy when used with the same feature set.


\end{document}
